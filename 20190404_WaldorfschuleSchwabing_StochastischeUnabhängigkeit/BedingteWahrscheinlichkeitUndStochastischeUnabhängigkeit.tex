\documentclass{article}
% Damit die Verwendung der deutschen Sprache nicht ganz so umst\"andlich wird,
% sollte man die folgenden Pakete einbinden: 
\usepackage[utf8]{inputenc}% erm\"oglich die direkte Eingabe der Umlaute 


\usepackage{amsmath}
\usepackage[T1]{fontenc} % das Trennen der Umlaute
\usepackage[ngerman]{babel} % hiermit werden deutsche Bezeichnungen genutzt und 
                     % die W\"orter werden anhand der neue Rechtschreibung 
		     % automatisch getrennt.  
\usepackage{graphicx}
\usepackage{lmodern}

\usepackage{amsthm}
\newtheorem{theorem}{Satz}

\usepackage{titlesec}
\titleformat{\subsection}{\large\bfseries\sffamily}{}{0pt}{Aufgabe \thesubsection:\quad}

\newcommand{\exercise}[1]{\subsection{#1}}

\begin{document}
\setcounter{section}{1}\setcounter{subsection}{0}

\textbf{Bedingte Wahrscheinlichkeit und stochastische Unabh\"angigkeit}

Du weißt bereits: Zwei Ereignisse $A$ und $B$ sind stochastisch unabh\"angig, falls die Gleichung

\begin{align}
\rule{6cm}{.4pt}\label{formelone}
\end{align}

gilt.

Problem: Schießt ein Fußballer F mit links, geht der Ball mit Wahrscheinlichkeit $\frac{1}{3}$ in's Tor. Schießt F mit rechts, geht der Ball auch mit Wahrscheinlichkeit $\frac{1}{3}$ in's Tor.
Sind nun die Ereignisse L=``Schuss mit links'' und T=``Schuss in's Tor'' stochastisch unabh\"angig?

L\"osungsversuch: Wir m\"ussen \"uberpr\"ufen, ob die Gleichung 
\begin{align}
\rule{6cm}{.4pt} \label{formeltwo}
\end{align}
gilt.
Das in $\eqref{formeltwo}$ auftauchende \rule{1cm}{.4pt} k\"onnen wir allerdings nicht berechnen. 

Wir brauchen also Werkzeuge, um die stochatische Unabh\"angigkeit auf andere Weise zu pr\"ufen.

\begin{theorem}
  Ist 
  \begin{align}
  P_A\left(B\right) = P\left(B\right) \label{formelthree}
  \end{align}, so sind $A$ und $B$ stochastisch unabh\"angig.
  \label{thm:foo}
\end{theorem}
\begin{proof}
Die Definition der bedingten Wahrscheinlichkeit in Gleichung $\eqref{formelthree}$ eingesetzt ergibt:
\vspace{0,5cm}
\begin{align}
\rule{6cm}{.4pt} \label{formelfour}
\end{align}
Gleichung $\eqref{formelfour}$ auf beiden Seiten mit \rule{1cm}{.4pt} multipliziert ergibt:
\begin{align}
\rule{6cm}{.4pt} \notag
\end{align}
Also sind $A$ und $B$ stochastisch unabh\"angig.
\end{proof}

2.L\"osungsversuch: Wir versuchen Satz~\ref{thm:foo} zu verwenden. Also m\"ussen wir pr\"ufen, ob die Gleichung 
\begin{align}
\rule{6cm}{.4pt} \notag
\end{align}
gilt.

$$P_L\left(T\right) = \rule{1cm}{.4pt}$$

\begin{align}
P\left(T\right) &= \rule{6cm}{.4pt} & \text{(1.Pfadregel}) \notag \\
&= \rule{6cm}{.4pt} &\text{(2.Pfadregel)} \notag \\
&= \rule{6cm}{.4pt} &\text{(aus der Angabe)} \notag \\
&= \rule{6cm}{.4pt} &\text{(Distributivgesetz)} \notag \\
&= \rule{1cm}{.4pt} \notag
\end{align}
Also sind $L$ und $T$ stochastisch \rule{6cm}{.4pt} .

Wir wollen noch ein anderes Werkzeug lernen, mit dem man stochastische Unabh\"angigkeit nachweisen kann.
\begin{theorem}
  Ist 
  \begin{align}
  P_A\left(B\right) = P_{\overline{A}}\left(B\right) \label{formelfive}
  \end{align}, so sind $A$ und $B$ stochastisch unabh\"angig.
  \label{thm:bar}
\end{theorem}
\begin{proof}
 \begin{align}
 P\left(B\right) &= \rule{6cm}{.4pt} & \text{(1.Pfadregel}) \notag \\
&= \rule{6cm}{.4pt} &\text{(2.Pfadregel)} \notag \\
&= \rule{6cm}{.4pt} &\text{(wegen} \eqref{formelfive} \text{)} \notag \\
&= \rule{6cm}{.4pt} &\text{(Distributivgesetz)} \notag \\
&= \rule{1cm}{.4pt} \notag
 \end{align}
 Also ist $P\left(B\right)=$ \rule{1cm}{.4pt}. Damit folgt mit Satz \rule{1cm}{.4pt}, dass $A$ und $B$ stochastisch unabh\"angig sind.
 
\end{proof}

3.L\"osungsversuch: Wir versuchen nun Satz \rule{1cm}{.4pt} zu verwenden.
$$ \rule{2cm}{.4pt} = \frac{1}{3} = \rule{2cm}{.4pt} $$.
Also sind nach Satz \rule{1cm}{.4pt} $A$ und $B$ stochastisch \rule{6cm}{.4pt}.

\end{document}
