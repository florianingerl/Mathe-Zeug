\documentclass{article}
% Damit die Verwendung der deutschen Sprache nicht ganz so umst\"andlich wird,
% sollte man die folgenden Pakete einbinden: 
\usepackage[utf8]{inputenc}% erm\"oglich die direkte Eingabe der Umlaute 
\usepackage[T1]{fontenc} % das Trennen der Umlaute
\usepackage{ngerman} % hiermit werden deutsche Bezeichnungen genutzt und 
                     % die W\"orter werden anhand der neue Rechtschreibung 
		     % automatisch getrennt.  
\usepackage{graphicx}

\usepackage{titlesec}
\titleformat{\subsection}{\large\bfseries\sffamily}{}{0pt}{Aufgabe \thesubsection:\quad}

\newcommand{\exercise}[1]{\subsection{#1}}

\begin{document}
\setcounter{section}{1}\setcounter{subsection}{0}
\exercise{}
Bei einer Sonnenhöhe von $35,2^\circ$ wirft eine frei stehende Tanne einen $18,3$ m langen Schatten. Wie hoch ist der Baum?

\exercise{}
Eine $2,25$ m hohe Stange wirft einen Schatten von $3,15$ m L\"ange. Unter welchem Winkel fallen die Sonnenstrahlen ein?

\exercise{}
Eine Treppe soll einen Neigungswinkel von $32^\circ$ bekommen. Die Stufen sind $15$ cm hoch. Berechne die Stufentiefe.

\exercise{}
Bestimme den Steigungswinkel $\alpha$ (Schnittwinkel $\alpha$ mit der $x$-Achse) der Geraden g:
\begin{itemize}
\item[a)] $g: y = 2x-1$
\item[b)] $g: y = \frac{2}{3} x + 5$
\end{itemize}

\exercise{}
Die Steigung einer Autobahn darf in L\"angsrichtung h\"ochstens $4,3\%$ betragen. Welchem Winkel $\alpha$ gegen die Horizontale entspricht dies?


\end{document}
