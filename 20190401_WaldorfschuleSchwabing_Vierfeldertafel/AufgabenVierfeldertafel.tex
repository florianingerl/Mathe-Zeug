\documentclass{article}
% Damit die Verwendung der deutschen Sprache nicht ganz so umst\"andlich wird,
% sollte man die folgenden Pakete einbinden: 
\usepackage[utf8]{inputenc}% erm\"oglich die direkte Eingabe der Umlaute 
\usepackage[T1]{fontenc} % das Trennen der Umlaute
\usepackage{ngerman} % hiermit werden deutsche Bezeichnungen genutzt und 
                     % die W\"orter werden anhand der neue Rechtschreibung 
		     % automatisch getrennt.  
\usepackage{graphicx}

\usepackage{titlesec}
\titleformat{\subsection}{\large\bfseries\sffamily}{}{0pt}{Aufgabe \thesubsection:\quad}

\newcommand{\exercise}[1]{\subsection{#1}}

\begin{document}
\setcounter{section}{1}\setcounter{subsection}{0}

\textbf{1.} Der Sch\"uler S f\"ahrt $50\%$ der Schultage mit dem Bus. In $70\%$ dieser F\"alle kommt er p\"unktlich zur Schule. Durchschnittlich kommt er aber nur an $60\%$ der Schultage p\"unktlich an. Heute kommt S p\"unktlich zur Schule. Mit welcher Wahrscheinlichkeit hat er den Bus benutzt?

\textbf{2.} Von $100$ Sch\"ussen des Fußballers F sind $60$ mit dem linken Fuß getreten. Von den Sch\"ussen mit dem linken Fuß gehen $20\%$ in das Tor. Insgesamt gehen aber nur $10\%$ der Sch\"usse von F in das Tor. Im letzten Spiel hat F ein Tor geschossen. Mit welcher Wahrscheinlichkeit tat er dies mit dem rechten Fuß?

\textbf{3.} Der Tennisspieler T macht $30\%$ seiner Aufschl\"age von unten. Von diesen Aufschl\"agen kommen dann $90\%$ in's Feld. Insgesamt kommen aber nur $60\%$ der Aufschl\"age von T in's Feld. Der letzte Aufschlag von T landete im Feld. Mit welcher Wahrscheinlichkeit war das ein Aufschlag von oben?

\textbf{4.} In Beutel $1$ sind $3$ schwarze und $4$ weiße Kugeln. In Beutel $2$ sind $5$ schwarze und $4$ weiße Kugeln. Der Spieler S wirft zun\"achst eine M\"unze, die mit einer Wahrscheinlichkeit von $\frac{1}{3}$ Zahl zeigt. Bei Zahl zieht er anschließend eine Kugel aus Beutel $1$, bei Kopf zieht der eine Kugel aus Beutel $2$. Nun wirft S die M\"unze und zieht anschließend aus einem der Beutel eine weiße Kugel. Mit welcher Wahrscheinlichkeit stammt diese aus Beutel $1$?


\end{document}
