\documentclass{article}
% Damit die Verwendung der deutschen Sprache nicht ganz so umst\"andlich wird,
% sollte man die folgenden Pakete einbinden: 
\usepackage[utf8]{inputenc}% erm\"oglich die direkte Eingabe der Umlaute 
\usepackage[T1]{fontenc} % das Trennen der Umlaute
\usepackage{ngerman} % hiermit werden deutsche Bezeichnungen genutzt und 
                     % die W\"orter werden anhand der neue Rechtschreibung 
		     % automatisch getrennt.  
\usepackage{graphicx}
\usepackage{amsmath}

\usepackage{titlesec}
\titleformat{\subsection}{\large\bfseries\sffamily}{}{0pt}{Aufgabe \thesubsection:\quad}

\newcommand{\exercise}[1]{\subsection{#1}}

\begin{document}
\setcounter{section}{1}\setcounter{subsection}{0}



\textbf{Aufgabe 1)} Von $100$ Sch\"ussen des Fußballers F sind $60$ mit dem linken Fuß getreten. Von den Sch\"ussen mit dem linken Fuß gehen $20\%$ in das Tor. Insgesamt gehen aber nur $15\%$ der Sch\"usse von F in das Tor. Im letzten Spiel hat F ein Tor geschossen. Mit welcher Wahrscheinlichkeit tat er dies mit dem rechten Fuß?

L\"osung:
$L=$``Schuss mit links''
$R=$``Schuss mit rechts''
$T=$``Tor''

Gegeben ist $P\left(L\right) = 0,6$, $P_L\left(T\right)=0,2$ und $P\left(T\right)=0,15$.

Gesucht ist $P_T\left(R\right)$.

Aus den Angaben l\"asst sich außerdem bestimmen:
\begin{align} P\left(L \cap T\right) &= P\left(L\right) \cdot P_L\left(T\right) \notag \\
&= 0,6 \cdot 0,2 \notag \\
&= 0,12 \notag
\end{align}

Die bisher bekannten Wahrscheinlichkeiten werden in die Vierfeldertafel eingetragen und diese wird vervollst\"andigt.

\begin{center}


\renewcommand{\arraystretch}{1.7}
 \begin{tabular}{c|c|c|c}
 
 & $T$ & $\overline{T}$ & 
 \\
 \hline
 $L$ & $0,12$ & $0,48$ & $0,6$ \\
 \hline
 $R$ & $0,03$ & $0,37$ & $0,4$  \\ 
 \hline
 & $0,15$ & $0,85$ & $1$
 \end{tabular}
 
 \end{center}
 
 \begin{align}
 P_T\left(R\right) &= \frac{P\left(R \cap T\right)}{P\left(T\right)} \notag \\
 &= \frac{0,03}{0,15} \notag \\
 &= \frac{1}{5} \notag \\
 &= 20\% \notag
 \end{align}
 
 Antwort: Mit einer Wahrscheinlichkeit von $20\%$ hat F das Tor mit rechts geschossen.
 
 \textbf{Aufgabe 2)} Der Sch\"uler S f\"ahrt $50\%$ der Schultage mit dem Bus. In $70\%$ dieser F\"alle kommt er p\"unktlich zur Schule. Durchschnittlich kommt er aber nur an $60\%$ der Schultage p\"unktlich an. Heute kommt S p\"unktlich zur Schule. Mit welcher Wahrscheinlichkeit hat er den Bus benutzt?

\textbf{Aufgabe 3)} Der Tennisspieler T macht $30\%$ seiner Aufschl\"age von unten. Von diesen Aufschl\"agen kommen dann $90\%$ in's Feld. Insgesamt kommen aber nur $60\%$ der Aufschl\"age von T in's Feld. Der letzte Aufschlag von T landete im Feld. Mit welcher Wahrscheinlichkeit war das ein Aufschlag von oben?

\textbf{Aufgabe 4)} Der Vogel V steht an $80\%$ der Tage fr\"uh auf. In $90\%$ dieser F\"alle f\"angt er sich einen Wurm zum Fr\"uhst\"uck. Insgesamt bekommt V aber nur an $75\%$ der Tage ein Fr\"uhst\"uck. Heute bekam V zum Fr\"uhst\"uck nichts. Mit welcher Wahrscheinlichkeit ist V fr\"uh aufgestanden? 


\end{document}
