\documentclass[11pt]{article}
\usepackage[a4paper,left=3cm,right=3cm,top=2cm,bottom=4cm,bindingoffset=5mm]{geometry}
\usepackage[utf8]{inputenc}
\usepackage{enumitem}
\usepackage{amsmath}


\title{ Lage-und Formver\"anderung von Funktionen }


\begin{document}

\maketitle

\textbf{Aufgabe 1} Ueberpr\"ufen Sie, ob der Punkt $P$ auf dem Graphen $G_f$ der Funktion $f$ liegt!
\begin{enumerate}[label=(\alph*)]
\item $f(x)=x^2+1, P(2|4)$ \\
L\"osungsbeispiel: $f(2) = 2^2+1=4+1=5 \neq 4 \Rightarrow P$ liegt nicht auf $G_f$ 
\item $f(x)=x^2+1, P(-2|5)$ 
\item $f(x)=sin\left( x + \frac{\pi}{2} \right), P\left( \frac{\pi}{2} | -1 \right)$
\item $f(x)=sin\left( x + \frac{\pi}{2} \right), P\left( \pi | 1 \right)$
\item $f(x)=sin\left( x + \frac{\pi}{2} \right), P\left( -\frac{\pi}{2} | 0 \right)$
\end{enumerate}

\textbf{Aufgabe 2} Es sei $P(2|5)$ ein Punkt auf dem Graphen $G_f$ der Funktion $f$ (d.h. $f(2)=5$). \\
Gib die Koordinaten des Punktes $Q$ an, der auf dem Graphen $G_g$ der Funktion $g$ liegt, falls
\begin{enumerate}[label=(\alph*)]
\item $G_g$ aus $G_f$ durch Verschiebung um $2$ in $x$-Richtung entsteht. \\
L\"osungsbeispiel: $Q(2+2|5)$
\item $G_g$ aus $G_f$ durch Verschiebung um $-1$ in $x$-Richtung entsteht.
\item $G_g$ aus $G_f$ durch Stauchung um den Faktor $\frac{1}{3}$ in $x$-Richtung entsteht. 
\item $G_g$ aus $G_f$ durch Streckung um den Faktor $3$ in $x$-Richtung entsteht.
\end{enumerate}

\textbf{Aufgabe 3} Gebe den Funktionsterm $g(x)$ zu dem Graphen an, der aus dem Graphen von $f$ durch die folgenden Ver\"anderungen entsteht. \\
\"Uberpr\"ufe dein Ergebnis wie im L\"osungsbeispiel.

\begin{enumerate}[label=(\alph*)]
\item $f(x)=\sin(x)$, Verschiebung um $3$ in $x$-Richtung \\
L\"osungsbeispiel: \\
$g(x)=\sin(x-3)$ \\
Probe: \\
Der Punkt $P( \frac{\pi}{2} | \underbrace{ f\left( \frac{\pi}{2} \right) }_{ = \sin\left( \frac{\pi}{2} \right) = 1  })$ liegt auf dem Graphen von $f$. \\
Der Punkt $Q\left( \frac{\pi}{2} + 3 |  1 \right)$ sollte auf dem Graphen von $g$ liegen. \\
Es sollte also $g\left( \frac{\pi}{2} + 3 \right) =  1 $ gelten. \\
Wir pr\"ufen das nach: \\
$g\left( \frac{\pi}{2} + 3 \right) = \sin\left( \left( \frac{\pi}{2} + 3 \right) - 3 \right) = \sin\left( \frac{\pi}{2} \right) = f\left( \frac{\pi}{2} \right) = 1  $
\item $f(x)= \sin(x)$, Verschiebung um $2$ in $x$-Richtung
\item $f(x) = \cos(x)$, Verschiebung um $-1$ in $x$-Richtung
\item $f(x)= \sin(x)$, Stauchung um den Faktor $\frac{1}{3}$ in $x$-Richtung
\item $f(x) = \cos(x)$, Streckung um den Faktor $3$ in $x$-Richtung
\item $f(x) = 2 \cdot \sin(x) + 3 \cdot \cos(x) $, Verschiebung um $4$ in $x$-Richtung
\item $f(x) = \sin(x) + x^2$, Verschiebung um $-1$ in $x$-Richtung
\end{enumerate}

\end{document}
